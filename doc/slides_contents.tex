%%%%%%%%%%%%%%%%%%%%%%%%%%%%%%%%%%%%%%%%%%%%%%%%%%%%%%%%%%%%%%%
%%%%%%%%%%%%%%%%%%%%%%%%%%%%%%%%%%%%%%%%%%%%%%%%%%%%%% Oversigt
\section{Introduktion}
\begin{frame}
    \vspace{25mm}
    \begin{center}
        \Huge{Part 0:\\Introduktion}
    \end{center}
\end{frame}

\subsection{Overblik}
\begin{frame}
    \frametitle{Overblik}
    \begin{center}
      \scalebox{1.6}{\includeSVG{pubsub_flow}}
    \end{center}
    \vspace{8mm}
    Vi skal lave et \textsl{dashboard}!
\end{frame}

\subsection{Repositorie}
\begin{frame}
    \frametitle{Repositorie}
    \begin{center}
      \scalebox{0.8}{\includeSVG{repo-barcode}}
      \url{https://github.com/aslakjohansen/sdu-sest-student-visit}
    \end{center}
\end{frame}

\subsection{Tekst Editor}
\begin{frame}
    \frametitle{Tekst Editor}
    Vi skal skrive noget kode. Det gøres i en \textsl{tekst editor}.
    
    \vspace{1cm}
    \begin{center}
        \begin{tabular}{l|c|c|c|}
            \rowcolor{blue!25} \cellcolor{blue!0}
                                  & \textbf{Windows} & \textbf{OSX} & \textbf{Linux} \\
            \hline
            \rowcolor{blue!10} \cellcolor{blue!25}
            \textbf{Installeret:} & Notepad          & TextEdit     & \ldots \\
            \hline
            \rowcolor{blue!10} \cellcolor{blue!25}
            \textbf{Bedre valg:}  & Sublime Text     & Sublime Text & Gedit/Kate/Sublime Text \\
            \hline
        \end{tabular}
    \end{center}
    \vspace{1cm}
    
    Sublime Text download: \url{https://www.sublimetext.com/download}
\end{frame}

%%%%%%%%%%%%%%%%%%%%%%%%%%%%%%%%%%%%%%%%%%%%%%%%%%%%%%%%%%%%%%%
%%%%%%%%%%%%%%%%%%%%%%%%%%%%%%%%%%%%%%%%%%%%%%% HTML Dokumenter
\section{HTML Dokumenter}
\begin{frame}
    \vspace{25mm}
    \begin{center}
        \Huge{Part 1:\\HTML Dokumenter}
    \end{center}
\end{frame}

\subsection{Arbejde med HTML Filer}
\begin{frame}
    \frametitle{Arbejde med HTML Filer}
    \vspace{20mm}
    \begin{center}
      \scalebox{1.6}{\includeSVG{html_flow}}
    \end{center}
\end{frame}

\subsection{Hello, World}
\begin{frame}
    \frametitle{Hello, World}
    \inputminted{html}{../src/frontend/iteration1_html_hello/index.html}
\end{frame}

\subsection{Anatomien af et Tag}
\begin{frame}
    \frametitle{Anatomien af et Tag}
    \begin{itemize}
      \item Begyndelse of afslutning.
      \item Navn.
      \item Shorthand hvis indhold er tomt.
      \item parametre.
    \end{itemize}
\end{frame}

\subsection{Eksempel}
\begin{frame}
    \frametitle{Eksempel}
    \inputminted[breaklines=true]{html}{../src/frontend/iteration1_html/index.html}
\end{frame}

\subsection{Struktur}
\begin{frame}
    \frametitle{Struktur}
    \begin{itemize}
      \item 
    \end{itemize}
\end{frame}

%%%%%%%%%%%%%%%%%%%%%%%%%%%%%%%%%%%%%%%%%%%%%%%%%%%%%%%%%%%%%%%
%%%%%%%%%%%%%%%%%%%%%%%%%%%%%%%%%%%%%%%%%% Logik med JavaScript
\section{Logik med JavaScript}
\begin{frame}
    \vspace{25mm}
    \begin{center}
        \Huge{Part 2:\\Logik med JavaScript}
    \end{center}
\end{frame}

\subsection{JavaScript?}
\begin{frame}
    \frametitle{JavaScript?}
    Et programmeringssprog der
    \begin{itemize}
      \item primært er designet til at kunne afvikles i en browser.
      \item kan manipulere den HTML datastruktur der vises i browseren.
      \item kan reagere på handlinger på hjemmesiden.
      \item kan interagere med omverdenen.
      \item kan udvides igennem moduler.
    \end{itemize}
\end{frame}

\subsection{Logik}
\begin{frame}
    \frametitle{Logik}
    \begin{itemize}
      \item 
    \end{itemize}
\end{frame}

\subsection{Konsollen}
\begin{frame}
    \frametitle{Konsollen}
    \begin{itemize}
      \item 
    \end{itemize}
\end{frame}

\subsubsection{Firefox}
\begin{frame}
    \frametitle{Firefox}
    \begin{itemize}
      \item 
    \end{itemize}
\end{frame}

\subsubsection{Chrome og Chromium}
\begin{frame}
    \frametitle{Chrome og Chromium}
    \begin{itemize}
      \item 
    \end{itemize}
\end{frame}

%%%%%%%%%%%%%%%%%%%%%%%%%%%%%%%%%%%%%%%%%%%%%%%%%%%%%%%%%%%%%%%
%%%%%%%%%%%%%%%%%%%%%%%%%%%%%%%%%%%%%%%%%%%%%%%%%% Tekststrenge
\section{Tekststrenge}
\begin{frame}
    \vspace{25mm}
    \begin{center}
        \Huge{Part 3:\\Tekststrenge}
    \end{center}
\end{frame}

\subsection{Tekststrenge?}
\begin{frame}
    \frametitle{Tekststrenge?}
    \begin{itemize}
      \item 
    \end{itemize}
\end{frame}

\subsection{Datastrukturer}
\begin{frame}
    \frametitle{Datastrukturer}
    \begin{itemize}
      \item 
    \end{itemize}
\end{frame}

\subsection{JSON}
\begin{frame}
    \frametitle{JSON}
    \begin{itemize}
      \item 
    \end{itemize}
\end{frame}

%%%%%%%%%%%%%%%%%%%%%%%%%%%%%%%%%%%%%%%%%%%%%%%%%%%%%%%%%%%%%%%
%%%%%%%%%%%%%%%%%%%%%%%%%%%%%%%%%%%%%%%%%%%%% Publish Subscribe
\section{Publish Subscribe}
\begin{frame}
    \vspace{25mm}
    \begin{center}
        \Huge{Part 4:\\Data med Publish Subscribe}
    \end{center}
\end{frame}

\begin{frame}
    \frametitle{Publish Subscribe}
    \vspace{10mm}
    \begin{center}
      \scalebox{1.6}{\includeSVG{pubsub_flow}}
    \end{center}
\end{frame}

%%%%%%%%%%%%%%%%%%%%%%%%%%%%%%%%%%%%%%%%%%%%%%%%%%%%%%%%%%%%%%%
%%%%%%%%%%%%%%%%%%%%%%%%%%%%%%%%%%%%%%%%%%%%%%%%%%%%%%%%%% Test
\section{Test}
\begin{frame}
    \vspace{25mm}
    \begin{center}
        \Huge{Part 5:\\Test}
    \end{center}
\end{frame}

%%%%%%%%%%%%%%%%%%%%%%%%%%%%%%%%%%%%%%%%%%%%%%%%%%%%%%%%%%%%%%%
%%%%%%%%%%%%%%%%%%%%%%%%%%%%%%%%%%%%%%%%%%%%%%%%%%%%%%%%% Plots
\section{Plots}
\begin{frame}
    \vspace{25mm}
    \begin{center}
        \Huge{Part 6:\\Plots}
    \end{center}
\end{frame}

%%%%%%%%%%%%%%%%%%%%%%%%%%%%%%%%%%%%%%%%%%%%%%%%%%%%%%%%%%%%%%%
%%%%%%%%%%%%%%%%%%%%%%%%%%%%%%%%%%%%%%%%%%%%%%%%%%%%% Questions
\section{Questions}
\begin{frame}
        \frametitle{\textbf{Questions?}}
        \vspace{-15mm}
        \begin{center}
        \includegraphics[scale=0.4]{./figs/Boy-asking-question.pdf}
        \end{center}
        \vspace{-25mm}
        \scalebox{0.2}{\url{https://openclipart.org/detail/238687/boy-thinking-of-question}}
\end{frame}

