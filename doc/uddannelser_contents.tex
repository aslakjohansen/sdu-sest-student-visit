\newcommand{\componentA}{0.9451}
\newcommand{\componentB}{0.9804}
\newcommand{\componentC}{0.9961}

\definecolor{intro}{rgb}{\componentA, \componentC, \componentB}
\definecolor{sdu}{rgb}{  \componentA, \componentB, \componentC}
\definecolor{se}{rgb}{   \componentB, \componentC, \componentA}
\definecolor{st}{rgb}{   \componentB, \componentA, \componentC}

\newcommand{\bgcolor}{Purple} % this should be renewed below

%%%%%%%%%%%%%%%%%%%%%%%%%%%%%%%%%%%%%%%%%%%%%%%%%%%%%%%%%%%%%%%
%%%%%%%%%%%%%%%%%%%%%%%%%%%%%%%%%%%%%%%%%%%%%%%%%%%%%%%%%%%%%%%

% args: 1:prefix 2:semester 3:index_west 4:index_east 5:title 6:ects 7:style
\newcommand{\entry}[7]{
  \node[
    rectangle,
    draw=purple,
    anchor=south west,
    align=center,
    minimum height=\cellheight,
    minimum width=(#4-#3)*\cellwidth+((#4-#3)-1)*\cellspacing,
    #7
  ]
  (#1) at (
    [
      xshift=\xoffset+#3*\cellwidth+(#3-1)*\cellspacing,
      yshift=\yoffset+(#2-1)*\cellheight+(#2-2)*\cellspacing,
    ] semorigo)
  {
    #5\\\textcolor{orange}{#6 ECTS}
  }
}

\newcommand{\entryNormal}[6]{\entry{#1}{#2}{#3}{#4}{#5}{#6}{}}
\newcommand{\entryProject}[6]{\entry{#1}{#2}{#3}{#4}{#5}{#6}{fill=purple!10,}}
\newcommand{\entryElective}[6]{\entry{#1}{#2}{#3}{#4}{#5}{#6}{pattern=north east lines, pattern color=purple!10, even odd rule,}}
\newcommand{\entryNormalMsc}[6]{\entry{#1}{#2}{#3}{#4}{#5}{#6}{draw=blue}}
\newcommand{\entryProjectMsc}[6]{\entry{#1}{#2}{#3}{#4}{#5}{#6}{draw=blue,fill=blue!10,}}
\newcommand{\entryElectiveMsc}[6]{\entry{#1}{#2}{#3}{#4}{#5}{#6}{draw=blue,pattern=north east lines, pattern color=blue!10, even odd rule,}}

%%%%%%%%%%%%%%%%%%%%%%%%%%%%%%%%%%%%%%%%%%%%%%%%%%%%%%%%%%%%%%%
%%%%%%%%%%%%%%%%%%%%%%%%%%%%%%%%%%%%%%%%%%%%%%%%%% Introduktion

% background color pre
{
\setbeamercolor{background canvas}{bg=intro}
\renewcommand{\bgcolor}{intro}

\section{Introduktion}
\begin{frame}
    \vspace{25mm}
    \begin{center}
        \Huge{Part 1:\\Introduktion}
    \end{center}
\end{frame}

\subsection{Software Engineering vs Datalogi}
\begin{frame}[fragile]
  \frametitle{Introduktion \subpart{Software Engineering vs Datalogi}}
  \pause
  \vspace{-2mm}
  
  \scalebox{0.8}{
    \begin{tikzpicture}[remember picture,overlay]
      \newcommand{\blockwidth}[0]{3.8cm}
      \newcommand{\width}[0]{(4*\blockwidth+3*2mm)}
      \newcommand{\ewidth}[0]{(\width+0.8mm)}
      \newcommand{\height}[0]{9cm}
      \newcommand{\xoffset}[0]{6mm}
      
      \coordinate (origo) at (\xoffset,-\height);
      
      \tikzstyle{node}=[
        overlay,
        rectangle,
        draw=purple,
        anchor=center,
        align=center,
        thick,
        minimum width=\blockwidth,
        minimum height=1.2cm,
      ]
      \tikzstyle{title} = [
        anchor=south,
        align=center,
      ]
      \tikzstyle{edge}  = [
        very thick,
        >=stealth,
        draw=purple,
      ]
      
      % organization
      \only<2->{
        \node[label, anchor=north] (organization) at ([xshift=\ewidth/2] \xoffset,0) {\textbf{Organisation}};
      }
      
      % programming
      \only<3->{
        \node[node, anchor=center, minimum height=2.6cm,minimum width=\ewidth]
          (programming) at ({\xoffset+\ewidth/2}, -\height/2]) {\textbf{Programmering}};
      }
      
      % foundation
      \only<4->{
        \node[label, anchor=south] (foundation) at ([xshift=\ewidth/2] origo) {\textbf{Fundament}};
      }
      
      % CS boxes
      \only<5->{
        \node[node, anchor=north west]
          (compiler) at ([yshift=-2mm] programming.south west) {Kompiler};
        \draw[edge] (compiler.south) to [out=270,in=180] (foundation.west);
      }
      \only<6->{
        \node[node, anchor=north west]
          (os) at ([xshift=2mm] compiler.north east) {Styresystemer};
        \draw[edge] (os.south) to [out=270,in=90] ([xshift=-4mm] foundation.north);
      }
      \only<7->{
        \node[node, anchor=north west]
          (cpu) at ([xshift=2mm] os.north east) {CPU Arkitektur};
        \draw[edge] (cpu.south) to [out=270,in=90] ([xshift=4mm] foundation.north);
      }
      \only<8->{
        \node[node, anchor=north west]
          (dist) at ([xshift=2mm] cpu.north east) {Problemer Indenfor\\\scalebox{0.8}{Distribuerede Systemer}};
        \draw[edge] (dist.south) to [out=270,in=0] (foundation.east);
      }
      
      % SE boxes
      \only<9->{
        \node[node, anchor=south west]
          (pm) at ([yshift=2mm] programming.north west) {Project\\Management};
        \draw[edge] (pm.north) to [out=90,in=180] (organization.west);
      }
      \only<10->{
        \node[node, anchor=south west]
          (reqs) at ([xshift=2mm] pm.south east) {Requirement\\Analysis};
        \draw[edge] (reqs.north) to [out=90,in=270] ([xshift=-4mm] organization.south);
      }
      \only<11->{
        \node[node, anchor=south west]
          (devp) at ([xshift=2mm] reqs.south east) {Development\\Processes};
        \draw[edge] (devp.north) to [out=90,in=270] ([xshift=4mm] organization.south);
      }
      \only<12->{
        \node[node, anchor=south west]
          (business) at ([xshift=2mm] devp.south east) {Business\\Proposals};
        \draw[edge] (business.north) to [out=90,in=0] (organization.east);
      }
      
      % CS
      \only<13->{
        \draw[overlay,decorate,decoration={brace, mirror, amplitude=10pt, raise=5pt}] ([xshift=0mm] programming.north west) to node[black,midway,left=15pt] {\rotatebox{90}{Datalogi}} ([xshift=0mm] compiler.south west);
      }
      
      % SE
      \only<14->{
        \draw[overlay,decorate,decoration={brace, mirror, amplitude=10pt, raise=5pt}] ([xshift=0mm, yshift=2mm] dist.north east) to node[black,midway,right=15pt] {\rotatebox{90}{Software Engineering}} ([xshift=0mm] business.north east);
      }
    \end{tikzpicture}
  }
\end{frame}

\begin{frame}[fragile]
  \frametitle{Introduktion \subpart{Software Engineering vs Datalogi}}
  \vspace{-1mm}
  Forhold til teknologier:
  \begin{itemize}
    \descitem{Datalog} Producerer den
    \descitem{Softwareingeniør} Forbruger dem
  \end{itemize}
  
  \pause
  \vspace{2mm}
  At lave god software kræver en \ldots
  \begin{itemize}
    \descitem{Datalog} kreativ process
    \descitem{Softwareingeniør} forudsigelig process
  \end{itemize}
  
  \pause
  \vspace{2mm}
  Kan kode være kunst:
  \begin{itemize}
    \descitem{Datalog} Uenighed
    \descitem{Softwareingeniør} Nej
  \end{itemize}
  
  \pause
  \vspace{2mm}
  Er uddannelsen teoretisk eller praktisk?
  \begin{itemize}
    \descitem{Datalog} Begge dele
    \descitem{Softwareingeniør} Begge dele
  \end{itemize}
\end{frame}

\subsection{Andre Uddannelser på SDU}
\begin{frame}[fragile]
  \frametitle{Introduktion \subpart{Andre Uddannelser på SDU (1/2)}}
  \vspace{7mm}
  \textbf{Sundheds- og Velfærdsteknologi}
  \begin{itemize}
    \descitem{Ingeniør:} BSc og MSc
    \descitem{Domæne:} Anatomi, Bevægelse, Fysiologi, Medicinsk udstyr
    \item Programmering
  \end{itemize}
  
  \pause
  \vspace{9mm}
  \textbf{Spiludvikling og Læringsteknologi}
  \begin{itemize}
    \descitem{Ingeniør:} BSc \textcolor{purple}{(og MSc via Software Engineering)}
    \descitem{Domæne:} Læringsteori, Spilprogrammering, Virtual \& Augmented Reality, Robotik, Historiefortælling
    \item Programmering
  \end{itemize}
\end{frame}
\begin{frame}[fragile]
  \frametitle{Introduktion \subpart{Andre Uddannelser på SDU (2/2)}}
  \vspace{7mm}
  \textbf{Robotteknologi}
  \begin{itemize}
    \descitem{Ingeniør:} BSc (både diplom og civil) og MSc
    \descitem{Domæne:} Mekanik, Elektronik, Kinematik, Radio
    \item Programmering
  \end{itemize}
  
  \pause
  \vspace{9mm}
  \textbf{Drones and Autonomous Systems}
  \begin{itemize}
    \descitem{Ingeniør:} MSc (overbygning til robotteknologi)
    \descitem{Domæne:} Autonome systemer
    \item Programmering
  \end{itemize}
\end{frame}

\subsection{Software Engineering Processen}
\begin{frame}[fragile]
  \frametitle{Introduktion \subpart{Software Engineering Processen}}
  \pause
  \vspace{22mm}
  
  \scalebox{0.94}{
    \begin{tikzpicture}[remember picture,overlay]
      \newcommand{\xspacing}[0]{6mm}
      
      \coordinate (origo) at (2cm,0);
      
      \tikzstyle{node}=[
        overlay,
        rectangle,
        draw=purple,
        anchor=center,
        align=center,
        thick,
        inner sep=4mm,
        minimum height=1.2cm,
      ]
      \tikzstyle{edge}  = [
        thick,
        ->,
        >=stealth,
        draw=purple,
      ]
      
      % krav
      \only<2->{
        \node[node, anchor=north] (krav) at (origo) {Krav};
      }
      
      % ana
      \only<3->{
        \node[node, anchor=west] (ana) at ([xshift=\xspacing] krav.east) {Analyse};
        \draw[edge] (krav)--(ana);
      }
      
      % design
      \only<4->{
        \node[node, anchor=west] (design) at ([xshift=\xspacing] ana.east) {Design};
        \draw[edge] (ana)--(design);
      }
      
      % impl
      \only<5->{
        \node[node, anchor=west] (impl) at ([xshift=\xspacing] design.east) {Implementering};
        \draw[edge] (design)--(impl);
      }
      
      % eval
      \only<6->{
        \node[node, anchor=west] (eval) at ([xshift=\xspacing] impl.east) {Evaluering};
        \draw[edge] (impl)--(eval);
      }
      
      % iteration
      \only<7->{
        \draw[edge, dashed] (eval.south)
          |- ([yshift=-8mm] ana.south)
          -| (ana.south)
        ;
        \node[below] (iteration) at ([yshift=-8mm] $(eval.south)!.5!(ana.south)$) {\textsl{iteration}};
      }
      
      % ?
      \only<8->{
        \node[node, anchor=east, dashed] (question) at ([xshift=-\xspacing] krav.west) {?};
        \draw[edge, dashed] (question)--(krav);
      }
    \end{tikzpicture}
  }
\end{frame}

\subsection{Semesterprojekter}
\begin{frame}[fragile]
  \frametitle{Introduktion \subpart{Semesterprojekter}}
  \vspace{5mm}
  På TEK har alle (BSc) uddannelserne semesterprojekter på de fleste semestre.
  
  \pause
  \vspace{7mm}
  Et semesterprojekt:
  \begin{itemize}
    \item Binder semesterets øvrige kurser sammen igennem tværfagligt arbejde
    \item Praktisk problemorienteret arbejde
    \item Udføres i grupper
    \item Fylder 10 ECTS
    \item Rapportaflevering med mundtligt forsvar
  \end{itemize}
\end{frame}

% background color post
}

%%%%%%%%%%%%%%%%%%%%%%%%%%%%%%%%%%%%%%%%%%%%%%%%%%%%%%%%%%%%%%%
%%%%%%%%%%%%%%%%%%%%%%%%%%%%%%%%%%%%%%%% Den Syddanske Ingeniør

% background color pre
{
\setbeamercolor{background canvas}{bg=sdu}
\renewcommand{\bgcolor}{sdu}

\section{Den Syddanske Ingeniør}
\begin{frame}
  \vspace{25mm}
  \begin{center}
    \Huge{Part 2:\\Den Syddanske Ingeniør}
  \end{center}
\end{frame}

\subsection{Hvad er en Ingeniør?}
\begin{frame}[fragile]
  \frametitle{Hvad er en Ingeniør?}
  
  \newcommand{\scaletext}[1]{\scriptsize{#1}}
  
  \pause
  \vspace{0mm}
  \scaletext{
    Ingeniør, person, som er uddannet til at \only<2>{udføre teknisk forskning og udvikling samt til at løse tekniske opgaver}\only<3->{\textcolor{purple}{udføre teknisk forskning og udvikling samt til at løse tekniske opgaver}} og gennemføre projekter inden for bl.a. bygge- og anlægsarbejder, maskinkonstruktion, produktion og energi \only<-3>{under hensyntagen til mennesker, miljø og økonomi}\only<4->{\textcolor{purple}{under hensyntagen til mennesker, miljø og økonomi}}.
  }
  
  \vspace{3mm}
  \scaletext{
    \textbf{Etymologi:}
    Ordet ingeniør kommer af fransk ingénieur, afledt af mlat. ingenium '(krigs)maskine', af latin ingenium 'kløgt, begavelse, påfund'. 
  }
  
  \vspace{3mm}
  \scaletext{
    Oprindelig blev betegnelsen ingeniør kun brugt i militær sammenhæng om personer, der stod for fæstningsbyggeri, konstruktion af krigsmateriel osv. Fra 1760'erne begyndte man i England under den industrielle revolution at skelne mellem militære og civile ingeniører, hvor de sidste bl.a. stod for bygningen af store kanal- og vejanlæg. Den formelle uddannelse af ingeniører indledtes dog i Frankrig, hvor især oprettelsen i 1794 af École polytechnique med vægten lagt på et matematisk-naturvidenskabeligt grundlag dannede mønster for tilsvarende skoler i andre lande.
  }
  
  \vspace{3mm}
  \scaletext{
    I Danmark begyndte uddannelsen til ingeniør med H. C. Ørsteds grundlæggelse i 1829 af Den Polytekniske Læreanstalt, nuv. Danmarks Tekniske Universitet (DTU). Man kunne fra begyndelsen vælge mellem at blive kandidat i anvendt naturvidenskab (fra 1898 kaldt fabrikingeniør, fra 1948 kemiingeniør) og kandidat i mekanik (fra 1899 maskiningeniør). I 1857 begyndte uddannelsen til bygningsingeniør, en beskæftigelse, hvortil man hidtil havde benyttet bl.a. ingeniørofficerer. I 1903 kom den sidste af de traditionelle fire ingeniørretninger, elektroingeniør. Ved kongelig anordning af 1933 blev titlen civilingeniør, der tidligere kun havde været brugt om bygningsingeniører, forbeholdt som fællesbetegnelse for kandidaterne fra DTU.
  }
  
  \vspace{1mm}
  \scaletext{Kilde: \url{https://denstoredanske.lex.dk/ingeni\%C3\%B8r}}
\end{frame}

\subsection{Kompetencer}
\begin{frame}[fragile]
  \frametitle{Kompetencer}
  \vspace{-3mm}
%  En ingeniør fra SDU har:
  \begin{itemize}
    \pause
    \item Faglige kompetencer
    \pause
    \item Generelle kompetencer
      \begin{itemize}
        \pause
        \item Arbejde selvstændigt:
          \begin{itemize}
            \item Planlægge strategier for \textcolor{purple}{egen læring}
            \item Evaluere \textcolor{purple}{egen læring}
            \item \textcolor{purple}{Fordybe sig fagligt}
            \item \textcolor{purple}{Formulere og analysere} et problem på en struktureret måde
          \end{itemize}
        \pause
        \item Samarbejde:
          \begin{itemize}
            \item Arbejde \textcolor{purple}{tværfagligt} og sammen med personer med anden faglig og kulturel
baggrund
            \item \textcolor{purple}{Dokumentere og formidle sin viden og sine resultater såvel
mundtligt som skriftligt til forskellige målgrupper}
            \item Evaluere andres arbejde og give \textcolor{purple}{feedback}
            \item Arbejde projektorienteret og i teams
          \end{itemize}
        \pause
        \item Bringe sin viden, færdighed og kompetencer i \textcolor{purple}{praktisk anvendelse} og være \ldots
          \begin{itemize}
            \item \textcolor{purple}{åben} overfor nye problemstillinger og løsninger
            \item innovativ, kreativ og \textcolor{purple}{løsningsorienteret}
          \end{itemize}
      \end{itemize}
    \pause
    \item Kompetencer vedrørende:
      \begin{itemize}
        \item Internationalisering
        \item Virksomhedssamarbejde
        \item Innovation og entreprenørskab
      \end{itemize}
  \end{itemize}
\end{frame}

\subsection{Hvordan opnås kompetencerne?}
\begin{frame}[fragile]
  \frametitle{Hvordan opnås kompetencerne?}
  
  \pause
  \begin{tikzpicture}[remember picture, overlay]
    \newcommand{\size}[0]{25mm}
    
    \tikzstyle{edge}  = [
      very thick,
      >=stealth,
      draw=black,
    ]
    
    \node[circle, anchor=north east, minimum size=2*\size, draw, very thick] (big) at ([xshift=-12mm, yshift=-17mm] current page.north east) {};
    \node[circle, minimum size=1.6cm, draw, fill=\bgcolor!95!black, very thick] (student) at (big) {\small{Student}};
    \node[circle, minimum size=1.6cm, draw, fill=\bgcolor!95!black, very thick] (subject) at (big.north) {\small{Subject}};
    \node[circle, minimum size=1.6cm, draw, fill=\bgcolor!95!black, very thick] (team) at ([xshift=-\size*0.81418097053, yshift=-\size*0.58061118421] big) {\small{Team}};
    \node[circle, minimum size=1.6cm, draw, fill=\bgcolor!95!black, very thick] (project) at ([xshift=\size*0.81418097053, yshift=-\size*0.58061118421] big) {\small{Project}};
    \draw[<->, edge] (student) -- (subject);
    \draw[<->, edge] (student) -- (team);
    \draw[<->, edge] (student) -- (project);
  \end{tikzpicture}
  
  \vspace{0mm}
  \begin{itemize}
    \pause
    \item DSMI (udtales \quoted{dæsmi})
      \begin{itemize}
        \item Den Syddanske Model for Ingeniøruddannelser
      \end{itemize}
    \pause
    \item Sammenhængende tematiske semestre
      \begin{itemize}
        \item De \textcolor{purple}{første fire semestre} på ingeniøruddannelserne
          \begin{itemize}
            \item Sammenhængende og \textcolor{purple}{tematiserede}
            \item \textcolor{purple}{Tværfagligt} semesterprojekt, der har \textcolor{purple}{semester-}\\\textcolor{purple}{temaet} som overskrift
            \item Projektarbejdet en særlig stærk rolle
          \end{itemize}
        \item De første to semestre er næsten fælles for\\Software Engineering og Softwareteknologi$^\dagger$
          \begin{enumerate}
            \item Udvikling af Software Programmer
            \item Udvikling af Software Systemer / Udvikling af Cyber-Physical Software Systemer
          \end{enumerate}
        \item Hvert semester tilrettelægges af et semesterteam (undervisere og projektvejledere)
      \end{itemize}
    \pause
    \item \textcolor{purple}{Aktiverende undervisning} og aktiv læring i timeblokke (typisk med fire timer)
  \end{itemize}
\end{frame}

% background color post
}

%%%%%%%%%%%%%%%%%%%%%%%%%%%%%%%%%%%%%%%%%%%%%%%%%%%%%%%%%%%%%%%
%%%%%%%%%%%%%%%%%%%%%%%%%%%%%%%%%%%%%%%%%% Software Engineering

% background color pre
{
\setbeamercolor{background canvas}{bg=se}
\renewcommand{\bgcolor}{se}

\section{Software Engineering}
\begin{frame}
    \vspace{25mm}
    \begin{center}
        \Huge{Part 3:\\Software Engineering}
    \end{center}
\end{frame}

\subsection{Typiske Arbejdsopgaver som Færdiguddannet}
\begin{frame}[fragile]
  \frametitle{Software Engineering \subpart{Typiske Arbejdsopgaver som Færdiguddannet}}
  \vspace{1mm}
  Software ingeniører løser en bred vifte af opgaver:
  \begin{itemize}
    \item Programmering
    \item Projekt management
    \item Udlicitering
    \item Udarbejdelse af kravspecifikationer
    \item IT strategi
  \end{itemize}
  
  \pause
  \vspace{3mm}
  Kendetegnet for disse er, at der arbejdes med software i forhold til en organisation.
  
  \pause
  \vspace{3mm}
  Igennem sit virke vil software ingeniøren arbejde sammen med:
  \begin{itemize}
    \descitem{Dataloger} for at forstå de fundamentale konsekvenser af potentielle designs.
    \descitem{Domæneeksperter} for at forstå problemdomænet.
    \descitem{Beslutningstagere} for at kommunikere hvordan en løsning vil passe ind i organisationen.
  \end{itemize}
\end{frame}

\subsection{Studiets Opbygning}
\begin{frame}[fragile]
  \frametitle{Software Engineering \subpart{Studiets Opbygning}}
  \vspace{-2mm}
  
  \scalebox{0.47}{
    \begin{tikzpicture}[remember picture,overlay]
      \newcommand{\cellwidth}[0]{4.3cm}
      \newcommand{\cellheight}[0]{1.5cm}
      \newcommand{\cellspacing}[0]{2mm}
      \newcommand{\width}[0]{(4*\cellwidth+3*\cellspacing)}
      \newcommand{\ewidth}[0]{(\width+0.8mm)}
      \newcommand{\height}[0]{16cm}
      \newcommand{\xoffset}[0]{24mm}
      \newcommand{\labelpad}[0]{32mm}
      \newcommand{\yoffset}[0]{-\height}
      
      \coordinate (origo) at (\xoffset,\yoffset);
%      \coordinate (semorigo) at ([xshift=\xoffset] origo);
      \coordinate (semorigo) at (0,0);
      
      \tikzstyle{node}=[
        overlay,
        rectangle,
        draw=purple,
        anchor=center,
        align=center,
        thick,
        minimum width=\blockwidth,
        minimum height=1.2cm,
      ]
      \tikzstyle{title} = [
        anchor=south,
        align=center,
      ]
      \tikzstyle{edge}  = [
        very thick,
        >=stealth,
        draw=purple,
      ]
      
      % sem1
      \only<2->{
        \entryNormal{cos}{1}{0}{1}{\textbf{Computersystemer}\\}{5};
        \entryNormal{mat}{1}{1}{2}{\textbf{Matematik for}\\\textbf{Ingeniører}}{5};
        \entryNormal{oop}{1}{2}{4}{\textbf{Intro til Objektorienteret Programmering}\\}{10};
        \entryProject{semI}{1}{4}{6}{\textbf{Semesterprojekt:}\\\textbf{Udvikling af Softwareprogrammer}}{10};
        \node[align=left, minimum width=\labelpad, anchor=east] (labI) at (cos.west) {\textbf{Semester 1}};
      }
      
      % sem2
      \only<3->{
        \entryNormal{seo}{2}{0}{2}{\textbf{Software Engineering}\\\textbf{og Organisation}}{10};
        \entryNormal{vop}{2}{2}{3}{\textbf{Videregående}\\\textbf{OOP}}{5};
        \entryNormal{md}{2}{3}{4}{\textbf{Data Management}\\}{5};
        \entryProject{semII}{2}{4}{6}{\textbf{Semesterprojekt:}\\\textbf{Udvikling af Softwaresystemer}}{10};
        \node[align=left, minimum width=\labelpad, anchor=east] (labII) at (seo.west) {\textbf{Semester 2}};
      }
      
      % sem3
      \only<4->{
        \entryNormal{intdesign}{3}{0}{1}{\textbf{Interaction}\\\textbf{Design}}{5};
        \entryNormal{statdata}{3}{1}{2}{\textbf{Statistik og}\\\textbf{Dataanalyse}}{5};
        \entryNormal{webtech}{3}{2}{3}{\textbf{Web}\\\textbf{Technologies}}{5};
        \entryNormal{osd}{3}{3}{4}{\textbf{Operativsystemer}\\\textbf{og Dist. Systemer}}{5};
        \entryProject{semIII}{3}{4}{6}{\textbf{Semesterprojekt:}\\\textbf{Interaktive Distribuerede Systemer}}{10};
        \node[align=left, minimum width=\labelpad, anchor=east] (labIII) at (intdesign.west) {\textbf{Semester 3}};
      }
      
      % sem4
      \only<5->{
        \entryNormal{ai}{4}{0}{1}{\textbf{Kunstig}\\\textbf{Intelligens}}{5};
        \entryNormal{component}{4}{1}{2}{\textbf{Komponentbaserede}\\\textbf{Systemer}}{5};
        \entryNormal{mathalgodat}{4}{2}{4}{\textbf{Diskret Matematik, Algoritmer}\\\textbf{og Datastrukturer}}{10};
        \entryProject{semIV}{4}{4}{6}{\textbf{Semesterprojekt:}\\\textbf{Intelligente Softwaresystemer}}{10};
        \node[align=left, minimum width=\labelpad, anchor=east] (labIV) at (ai.west) {\textbf{Semester 4}};
      }
      
      % sem5
      \only<6->{
        \entryNormal{security}{5}{0}{1}{\textbf{Cybersecurity}\\\textbf{}}{5};
        \entryNormal{mobile}{5}{1}{2}{\textbf{Mobile Software}\\\textbf{Development}}{5};
        \entryNormal{maint}{5}{2}{3}{\textbf{Software}\\\textbf{Maintenance}}{5};
        \entryElective{valgI}{5}{3}{4}{\textbf{Valgfag}\\\textbf{}}{5};
        \entryElective{valgII}{5}{4}{5}{\textbf{Valgfag}\\\textbf{}}{5};
        \entryElective{valgIII}{5}{5}{6}{\textbf{Valgfag}\\\textbf{}}{5};
        \node[align=left, minimum width=\labelpad, anchor=east] (labV) at (security.west) {\textbf{Semester 5}};
      }
      
      % sem6
      \only<7->{
        \entryNormal{arch}{6}{0}{1}{\textbf{Software}\\\textbf{Arkitektur}}{5};
        \entryNormal{teori}{6}{1}{2}{\textbf{Ingeniørfagets}\\\textbf{Videnskabsteori}}{3};
        \entryNormal{ledelse}{6}{2}{3}{\textbf{Project}\\\textbf{Management}}{7};
        \entryProject{bsc}{6}{3}{6}{\textbf{Bachelorprojekt}\\\textbf{}}{15};
        \node[align=left, minimum width=\labelpad, anchor=east] (labVI) at (arch.west) {\textbf{Semester 6}};
      }
      
      % bsc
      \only<8->{
        \draw[overlay,decorate,decoration={brace, mirror, amplitude=10pt, raise=5pt}] (semI.south east) to node[black,midway,right=15pt] {\rotatebox{90}{\textbf{Bachelor}}} (bsc.north east);
      }
      
      % sem7
      \only<9->{
        \entryNormalMsc{sci}{7}{0}{1}{\textbf{Scientific}\\\textbf{Methods}}{5};
        \entryNormalMsc{advmethod}{7}{1}{2}{\textbf{Advanced SE}\\\textbf{Methodologies}}{5};
        \entryElectiveMsc{valgIV}{7}{2}{4}{\textbf{Specialiserings}\\\textbf{Valgfag}}{10};
        \entryElectiveMsc{valgV}{7}{4}{6}{\textbf{Specialiserings}\\\textbf{Valgfag}}{10};
        \node[align=left, minimum width=\labelpad, anchor=east] (labVII) at (sci.west) {\textbf{Semester 7}};
      }
      
      % sem8
      \only<10->{
        \entryNormalMsc{engresearch}{8}{0}{2}{\textbf{Engineering Research}\\\textbf{in Software}}{10};
        \entryElectiveMsc{valgVI}{8}{2}{4}{\textbf{Specialiserings}\\\textbf{Valgfag}}{10};
        \entryElectiveMsc{valgVII}{8}{4}{6}{\textbf{Specialiserings}\\\textbf{Valgfag}}{10};
        \node[align=left, minimum width=\labelpad, anchor=east] (labIIX) at (engresearch.west) {\textbf{Semester 8}};
      }
      
      % sem9
      \only<11->{
        \entryNormalMsc{inno}{9}{0}{2}{\textbf{Experts in}\\\textbf{Team Innovation}}{10};
        \entryElectiveMsc{valgIIX}{9}{2}{6}{\textbf{Valgfag}\\\textbf{}}{20};
        \node[align=left, minimum width=\labelpad, anchor=east] (labIX) at (inno.west) {\textbf{Semester 9}};
      }
      
      % sem10
      \only<12->{
        \entryProjectMsc{msc}{10}{0}{6}{\textbf{MSc Projekt}\\\textbf{}}{30};
        \node[align=left, minimum width=\labelpad, anchor=east] (labX) at (msc.west) {\textbf{Semester 10}};
      }
      
      % msc
      \only<13->{
        \draw[overlay,decorate,decoration={brace, mirror, amplitude=10pt, raise=5pt}] (valgV.south east) to node[black,midway,right=15pt] {\rotatebox{90}{\textbf{Master / Kandidat}}} (msc.north east);
      }
    \end{tikzpicture}
  }
\end{frame}

\subsection{Bachelorprojekt}
\begin{frame}[fragile]
  \frametitle{Software Engineering \subpart{Bachelorprojekt}}
  \pause
  \vspace{3mm}
  Den første store opgave.
  
  \vspace{5mm}
  Typisk i grupper af 2-3.
  
  \vspace{5mm}
  Ofte i samarbejde med en virksomhed.
  
  \pause
  \vspace{5mm}
  \textbf{Størrelse:} 15 ECTS
  
  \pause
  \vspace{5mm}
  \textbf{Evaluering:} Rapport med mundtligt forsvar
\end{frame}

\subsection{Udlandsophold}
\begin{frame}[fragile]
  \frametitle{Software Engineering \subpart{Udlandsophold}}
  \pause
  \vspace{3mm}
  Der er mulighed for at tilbringe et semester i udlandet.
  
  \vspace{5mm}
  \textbf{Hvornår:} På 9. semester \pause (aka 3. semester af kandidaten)
  
  \vspace{5mm}
  Vi har som Danskere ret gode mugligheder for at tage på udlandsophold.
\end{frame}

\subsection{In-Company Projekt}
\begin{frame}[fragile]
  \frametitle{Software Engineering \subpart{In-Company Projekt}}
  \pause
  \vspace{3mm}
  På kandidaten er der mulighed for at tilbringe noget tid i en virksomhed.
  
  \vspace{5mm}
  Der er ikke tale om praktik, men om at man hjælper virksomheden med at løse en bestemt problematik.
  
  \vspace{5mm}
  \textbf{Hvornår:} På 9. semester
  
  \pause
  \vspace{5mm}
  \textbf{Størrelse:} 15 ECTS
  
  \pause
  \vspace{5mm}
  \textbf{Evaluering:} Rapport med mundtligt forsvar
\end{frame}

\subsection{Kandidatprojekt}
\begin{frame}[fragile]
  \frametitle{Software Engineering \subpart{Kandidatprojekt}}
  \pause
  \vspace{3mm}
  Den helt store opgave.
  
  \vspace{5mm}
  Typisk i grupper af 2.
  
  \vspace{5mm}
  I finder selv en problematik som I vil arbejde med, og en vejleder.
  
  \vspace{5mm}
  Kan være i samarbejde med en virksomhed.
  
  \pause
  \vspace{5mm}
  \textbf{Størrelse:}
  \begin{itemize}
    \descitem{30 ECTS} som udgangspunkt\pause, men
    \descitem{40 ECTS} som mulighed. Dette vil \say{koste} et valgfag på 10 ECTS på 9. semester.
  \end{itemize}
  
  \pause
  \vspace{5mm}
  \textbf{Evaluering:} Rapport med mundtligt forsvar
\end{frame}

% background color post
}

%%%%%%%%%%%%%%%%%%%%%%%%%%%%%%%%%%%%%%%%%%%%%%%%%%%%%%%%%%%%%%%
%%%%%%%%%%%%%%%%%%%%%%%%%%%%%%%%%%%%%%%%%%%% Softwareteknologi

% background color pre
{
\setbeamercolor{background canvas}{bg=st}
\renewcommand{\bgcolor}{st}

\section{Softwareteknologi}
\begin{frame}
    \vspace{25mm}
    \begin{center}
        \Huge{Part 4:\\Softwareteknologi}
    \end{center}
\end{frame}

\subsection{Studiet}
\begin{frame}[fragile]
  \frametitle{Softwareteknologi \subpart{Studiet}}
  \vspace{3mm}
  Softwareteknologer er Software Ingeniører, men \ldots
  \begin{itemize}
    \item Endnu mere \textcolor{purple}{fokus på anvendelser}.
    \item Uddannelsen er rettet mod en specifikke opgaver:
      \begin{itemize}
        \item Understøttelse af den digitale transformation ved hjælp af \textcolor{purple}{web grænseflader} og \textcolor{purple}{test}.
        \item At bruge digital teknologi til at skabe løsninger der \textcolor{purple}{understøtter arbejdsprocesser som tidligere har været papirbaserede}.
      \end{itemize}
  \end{itemize}
  
  \pause
  \vspace{5mm}
  \textbf{Uddannelsen er mere specialiseret:} \\
  Hvor software ingeniør studiet give brede kompetencer, har man på softwareteknologi studiet valgt at \textcolor{purple}{skære de emner fra som ikke er relevante for ovenstående problemstillinger mod at man kan gå mere i dybden med disse}.
\end{frame}

\subsection{Studiets Opbygning}
\begin{frame}[fragile]
  \frametitle{Softwareteknologi \subpart{Studiets Opbygning}}
  \vspace{-2mm}
  
  \scalebox{0.47}{
    \begin{tikzpicture}[remember picture,overlay]
      \newcommand{\cellwidth}[0]{4.3cm}
      \newcommand{\cellheight}[0]{1.5cm}
      \newcommand{\cellspacing}[0]{2mm}
      \newcommand{\width}[0]{(4*\cellwidth+3*\cellspacing)}
      \newcommand{\ewidth}[0]{(\width+0.8mm)}
      \newcommand{\height}[0]{13.5cm}
      \newcommand{\xoffset}[0]{24mm}
      \newcommand{\labelpad}[0]{32mm}
      \newcommand{\yoffset}[0]{-\height}
      
      \coordinate (origo) at (\xoffset,\yoffset);
%      \coordinate (semorigo) at ([xshift=\xoffset] origo);
      \coordinate (semorigo) at (0,0);
      
      \tikzstyle{node}=[
        overlay,
        rectangle,
        draw=purple,
        anchor=center,
        align=center,
        thick,
        minimum width=\blockwidth,
        minimum height=1.2cm,
      ]
      \tikzstyle{title} = [
        anchor=south,
        align=center,
      ]
      \tikzstyle{edge}  = [
        very thick,
        >=stealth,
        draw=purple,
      ]
      
      % sem1
      \only<2->{
        \entryNormal{cos}{1}{0}{1}{\textbf{Computersystemer}}{5};
        \entryNormal{sda}{1}{1}{2}{\textbf{Matematik for}\\\textbf{Ingeniører}}{5};
        \entryNormal{oop}{1}{2}{4}{\textbf{Intro til Objektorienteret Programmering}\\}{10};
        \entryProject{semI}{1}{4}{6}{\textbf{Semesterprojekt:}\\\textbf{Udvikling af Softwareprogrammer}}{10};
        \node[align=left, minimum width=\labelpad, anchor=east] (labI) at (cos.west) {\textbf{Semester 1}};
      }
      
      % sem2
      \only<3->{
        \entryNormal{seo}{2}{0}{2}{\textbf{Software Engineering og Modelling}\\\textbf{til Cyber-Physical Systemer}}{10};
        \entryNormal{vop}{2}{2}{3}{\textbf{Videregående}\\\textbf{OOP}}{5};
        \entryNormal{md}{2}{3}{4}{\textbf{Data Management}\\}{5};
        \entryProject{semII}{2}{4}{6}{\textbf{Semesterprojekt: Udvikling af}\\\textbf{Softwaresystemer med CPS Elementer}}{10};
        \node[align=left, minimum width=\labelpad, anchor=east] (labII) at (seo.west) {\textbf{Semester 2}};
      }
      
      % sem3
      \only<4->{
        \entryNormal{indust}{3}{0}{1}{\textbf{Industrielle}\\\scalebox{0.9}{\textbf{Cyper-Physical Systemer}}}{5};
        \entryNormal{statdata}{3}{1}{2}{\textbf{Statistik og}\\\textbf{Dataanalyse}}{5};
        \entryNormal{webtech}{3}{2}{3}{\textbf{Web}\\\textbf{Technologies}}{5};
        \entryNormal{osd}{3}{3}{4}{\textbf{Operativsystemer}\\\textbf{og Dist. Systemer}}{5};
        \entryProject{semIII}{3}{4}{6}{\textbf{Semesterprojekt: Distribuerede SW Systemer}\\\textbf{med Industrielle Elementer}}{10};
        \node[align=left, minimum width=\labelpad, anchor=east] (labIII) at (indust.west) {\textbf{Semester 3}};
      }
      
      % sem4
      \only<5->{
        \entryNormal{reliable}{4}{0}{1}{\textbf{Design af Pålidelige}\\\scalebox{0.9}{\textbf{Cyper-Physical Systemer}}}{5};
        \entryNormal{swtech}{4}{1}{2}{\textbf{Softwareteknologi i}\\\scalebox{0.9}{\textbf{Cyber-Physical Systemer}}}{7};
        \entryNormal{compo}{4}{2}{3}{\textbf{Komponentbaserede}\\\textbf{Systemer}}{5};
        \entryNormal{eng}{4}{3}{4}{\textbf{Ingeniørfagets}\\\textbf{Videnskabsteori}}{3};
        \entryProject{semIV}{4}{4}{6}{\textbf{Semesterprojekt: Pålidelige Komponentbaserede}\\\textbf{Cyber-Physical Systemer}}{10};
        \node[align=left, minimum width=\labelpad, anchor=east] (labIV) at (reliable.west) {\textbf{Semester 4}};
      }
      
      % sem5
      \only<6->{
        \entryNormal{algo}{5}{0}{1}{\textbf{Algoritmer og}\\\textbf{Datastrukturer}}{5};
        \entryElective{valgI}{5}{1}{2}{\textbf{Valgfag}\\\textbf{}}{5};
        \entryElective{valgII}{5}{2}{3}{\textbf{Valgfag}\\\textbf{}}{5};
        \entryElective{valgIII}{5}{3}{4}{\textbf{Valgfag}\\\textbf{}}{5};
        \entryNormal{exp}{5}{4}{6}{\textbf{Experts in Teams}\\\textbf{Innovation}}{10};
        \node[align=left, minimum width=\labelpad, anchor=east] (labV) at (algo.west) {\textbf{Semester 5}};
      }
      
      % sem6
      \only<7->{
        \entryProject{prak}{6}{0}{6}{\textbf{Ingeniørpraktik}\\\textbf{}}{30};
        \node[align=left, minimum width=\labelpad, anchor=east] (labVI) at (prak.west) {\textbf{Semester 6}};
      }
      
      % sem7
      \only<8->{
        \entryProject{prak}{7}{0}{6}{\textbf{Afgangsprojekt}\\\textbf{}}{30};
        \node[align=left, minimum width=\labelpad, anchor=east] (labVI) at (prak.west) {\textbf{Semester 7}};
      }
      
      % diplom
      \only<9->{
        \draw[overlay,decorate,decoration={brace, mirror, amplitude=10pt, raise=5pt}] (semI.south east) to node[black,midway,right=15pt] {\rotatebox{90}{\textbf{Diplom}}} (prak.north east);
      }
    \end{tikzpicture}
  }
\end{frame}

\subsection{Ingeniørpraktik}
\begin{frame}[fragile]
  \frametitle{Softwareteknologi \subpart{Ingeniørpraktik}}
  \vspace{3mm}
  Brug et halvt år i praktik hos en virksomhed.
  
  \vspace{5mm}
  Her får I mulighed for at:
  \begin{itemize}
    \item Bruge en delmængde af alt hvad I har lært.
    \item Se en virksomhed an.
  \end{itemize}
\end{frame}

\subsection{Afgangsprojekt}
\begin{frame}[fragile]
  \frametitle{Softwareteknologi \subpart{Afgangsprojekt}}
  \vspace{3mm}
  Denne uddannelses store opgave.
  
  \vspace{5mm}
  Typisk i grupper af 2.
  
  \vspace{5mm}
  Typisk i samarbejde med en virksomhed som én af gruppemedlemmerne kender fra sin praktik.
  
  \pause
  \vspace{5mm}
  \textbf{Evaluering:} Rapport med mundtligt forsvar.
\end{frame}

% background color post
}

%%%%%%%%%%%%%%%%%%%%%%%%%%%%%%%%%%%%%%%%%%%%%%%%%%%%%%%%%%%%%%%
%%%%%%%%%%%%%%%%%%%%%%%%%%%%%%%%%%%%%%%%%%%%%%%%%%%%% Questions
\section{Spørgsmål}
\begin{frame}
    \frametitle{\textbf{Spørgsmål?}}
    \vspace{-15mm}
    \begin{center}
        \includegraphics[scale=0.4]{./figs/Boy-asking-question.pdf}
    \end{center}
    \vspace{-25mm}
    \scalebox{0.2}{\url{https://openclipart.org/detail/238687/boy-thinking-of-question}}
\end{frame}

